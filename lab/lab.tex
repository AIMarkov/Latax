\documentclass{article}
%\usepackage{ctex}
\usepackage{cite}
%\usepackage[style=authoryear,backend=biber]{biblatex}
%\newcommand\degree{^\circle}%建立新的符号
\title{my article}
\date{\today}
\begin{document}
\maketitle


%两个数学符号期望和正态分布:
$\mathbb{E}$
$\mathcal{N}$



\section{introduction}


hello world\cite{sutton1998}
hello world\cite{sutton2017}
hello world\cite{Rummery1}
hello world
hello world
hello world
hello world
hello world
hello world
hello world
hello world
hello world
hello world
hello world
hello world
hello world
hello world

$f(x)=x^2$
%equation模式用于产生带有编号的公式符号

    \begin{equation}
    G_{t}=\sum_{{t}'=t}^{T}\gamma ^{{t}'-t}r_{{t}'}
    \end{equation}
my name chenhongming$$f(x)=x^2(12)$$%双doller符号表示另起一行来表示公式
\bibliographystyle{plain}%这里用会议自带的参考文献格式
\nocite{*}%显示所有reference包括没有引用的
\bibliography{Reference}
\end{document} 
